\chapter{Einleitung}

In diesem Versuch wird das Verhalten von Spin-$\tfrac{1}{2}$-Teilchen im äußeren Magnetfeld anhand der kernmagnetischen Relaxation untersucht. Legt man auf eine Probe zunächst ein starkes homogenes Magnetfeld an und bringt die Kernspins anschließend durch einen transversalen, rotierenden RF-Puls aus ihrem Gleichgewichtszustand, so lassen sich die zeitlichen Prozesse der Rückkehr in das Gleichgewicht experimentell beobachten. Diese Prozesse werden durch die longitudinale Relaxationszeit $T_1$ und die transversale Relaxationszeit $T_2$ beschrieben und bilden die Grundlage vieler Anwendungen der Kernspinresonanz (NMR), etwa der Magnetresonanztomographie.

Ziel des Versuchs ist es, die grundlegenden Eigenschaften der Kernspinresonanz experimentell nachzuvollziehen. Dazu werden zunächst nach erfolgreicher Justage des Aufbaus Rabi-Oszillationen aufgenommen, bevor im Hauptteil die Relaxationszeiten $T_1$, $T_2^\ast$ und $T_2$ mithilfe verschiedener Pulssequenzen (Sättigungs- und Polarisations-Zurückgewinnung, Hahn-Echo, Carr-Purcell und Meiboom-Gill) bestimmt werden.
