\chapter{Einleitung}
In diesem Versuch wird die kernmagnetische Relaxation von Proben untersucht. Dabei werden die Relaxationszeiten $T_1$ und $T_2$ mittels Puls-NMR-Messungen bestimmt. Die Messungen erfolgen an einer Probe, die in einem starken Magnetfeld platziert wird, um die Kernspins auszurichten. Durch gezielte Radiofrequenzpulse werden die Spins aus ihrer Gleichgewichtslage gebracht, und die anschließende Relaxation wird aufgezeichnet.