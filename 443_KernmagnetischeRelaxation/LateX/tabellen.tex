\chapter{Tabellen}
% chktex-file 1
% chktex-file 24
\begin{table}[ht]
    \centering
    \begin{tabular}{
        S[table-format=2.1]          % A_len
        S[table-format=3.1(3)]       % U_env pm dU_env
        S[table-format=5.1(3)]       % U_I pm dU_I
    }
    \toprule
    {$A_{\mathrm{len}}$ / $\mathrm{\mu s}$} &
    {$U_{\mathrm{env}}$ / mV} &
    {$U_I$ / mV} \\
    \midrule
0.5  & 277.5 \pm 50   &  37   \pm  4.0  \\
1.0  & 577.5 \pm 50   &  72   \pm  7.5  \\
1.5  & 727.5 \pm 50   & 107   \pm 11.0  \\
2.0  & 887.5 \pm 50   & 117   \pm 12.0  \\
2.5  & 947.5 \pm 50   & 132   \pm 13.5  \\
3.0  & 932.5 \pm 50   & 127   \pm 13.0  \\
3.5  & 887.5 \pm 50   & 122   \pm 12.5  \\
4.0  & 797.5 \pm 50   & 108   \pm 11.1  \\
4.5  & 777.5 \pm 50   &  97   \pm 10.0  \\
5.0  & 529.5 \pm 50   &  74   \pm  7.7  \\
5.5  & 342.5 \pm 50   &  49   \pm  5.2  \\
6.0  & 184.5 \pm 50   &  23   \pm  2.6  \\
6.5  & 177.5 \pm 50   &  -5   \pm  0.2  \\
7.0  & 239.5 \pm 50   & -30   \pm  2.7  \\
7.5  & 515.5 \pm 50   & -53   \pm  5.0  \\
8.0  & 662.5 \pm 50   & -73   \pm  7.0  \\
8.5  & 787.5 \pm 50   & -90   \pm  8.7  \\
9.0  & 872.5 \pm 50   & -104  \pm 10.1  \\
9.5  & 929.5 \pm 50   & -110  \pm 10.7  \\
10.0 & 917.5 \pm 50   & -110  \pm 10.7  \\
10.5 & 869.5 \pm 50   & -103  \pm 10.0  \\
11.0 & 757.5 \pm 50   &  -90  \pm  8.7  \\
11.5 & 607.5 \pm 50   &  -70  \pm  6.7  \\
12.0 & 377.5 \pm 50   &  -44  \pm  4.1  \\
    \bottomrule
    \end{tabular}
    \caption{Messwerte der Amplituden $U_{\mathrm{env}}$ und $U_I$ in Abhängigkeit der Pulsdauer $A_{\mathrm{len}}$.}
    \label{tab:rabi_raw}
\end{table}
%
% ====
%
\begin{table}[ht]
    \centering
    \begin{tabular}{
        S[table-format=3.1(2)]          % tau
        S[table-format=4.0(2)]          % U
    }
    \toprule
    {$\tau$ / ms} &
    {$U$ / mV}\\
    \midrule
 50.0 \pm 0.5 &  750 \pm 15 \\
 60.0 \pm 0.6 &  825 \pm 16 \\
 70.0 \pm 0.7 &  862 \pm 17 \\
 80.0 \pm 0.8 &  906 \pm 18 \\
 90.0 \pm 0.9 &  931 \pm 18 \\
100.0 \pm 1.0 &  962 \pm 19 \\
110.0 \pm 1.1 &  981 \pm 19 \\
120.0 \pm 1.2 & 1000 \pm 20 \\
130.0 \pm 1.3 & 1010 \pm 20 \\
140.0 \pm 1.4 & 1030 \pm 20 \\
150.0 \pm 1.5 & 1040 \pm 20 \\
160.0 \pm 1.6 & 1060 \pm 21 \\
170.0 \pm 1.7 & 1070 \pm 21 \\
180.0 \pm 1.8 & 1080 \pm 21 \\
190.0 \pm 1.9 & 1075 \pm 21 \\
200.0 \pm 2.0 & 1095 \pm 21 \\
250.0 \pm 2.5 & 1115 \pm 22 \\
300.0 \pm 3.0 & 1132 \pm 22 \\
350.0 \pm 3.5 & 1142 \pm 22 \\
400.0 \pm 4.0 & 1142 \pm 22 \\
450.0 \pm 4.5 & 1157 \pm 23 \\
500.0 \pm 5.0 & 1162 \pm 23 \\
    \bottomrule
    \end{tabular}
    \caption{Messwerte der Sättigungs-Zurückgewinn-Methode. Für eine variable Wartezeit $\tau$ wird die Amplitude $U$ des FID-Signals gemessen. Es wurden Fehler von $\Delta U = \qty{2}{\percent}$ und $\Delta \tau = \qty{1}{\percent}$ angenommen.}
    \label{tab:saettigung}
\end{table}
%
% ====
%
\begin{table}[ht]
    \centering
    \begin{tabular}{
        S[table-format=3.1(2)]          % tau
        S[table-format=5.0(2)]          % U
    }
    \toprule
    {$\tau$ / ms} &
    {$U$ / mV}\\
    \midrule
 10.0 \pm 0.1 & -720 \pm 50 \\
 20.0 \pm 0.2 & -390 \pm 50 \\
 30.0 \pm 0.3 & -117 \pm 50 \\
 40.0 \pm 0.4 &  162 \pm 50 \\
 50.0 \pm 0.5 &  315 \pm 50 \\
 60.0 \pm 0.6 &  440 \pm 50 \\
 70.0 \pm 0.7 &  542 \pm 50 \\
 80.0 \pm 0.8 &  672 \pm 50 \\
 90.0 \pm 0.9 &  692 \pm 50 \\
100.0 \pm 1.0 &  742 \pm 50 \\
110.0 \pm 1.1 &  777 \pm 50 \\
120.0 \pm 1.2 &  872 \pm 50 \\
130.0 \pm 1.3 &  875 \pm 50 \\
140.0 \pm 1.4 &  890 \pm 50 \\
150.0 \pm 1.5 &  917 \pm 50 \\
160.0 \pm 1.6 &  940 \pm 50 \\
170.0 \pm 1.7 &  955 \pm 50 \\
180.0 \pm 1.8 &  972 \pm 50 \\
190.0 \pm 1.9 &  987 \pm 50 \\
200.0 \pm 2.0 & 1000 \pm 50 \\
250.0 \pm 2.5 & 1052 \pm 50 \\
300.0 \pm 3.0 & 1080 \pm 50 \\
350.0 \pm 3.5 & 1110 \pm 50 \\
400.0 \pm 4.0 & 1125 \pm 50 \\
450.0 \pm 4.5 & 1137 \pm 50 \\
500.0 \pm 5.0 & 1150 \pm 50 \\
    \bottomrule
    \end{tabular}
    \caption{Messwerte der Polarisations-Zurückgewinn-Methode. Für eine variable Wartezeit $\tau$ wird die Amplitude $U$ des FID-Signals gemessen. Es wurden Fehler von $\Delta U = \qty{50}{\milli\volt}$ (aufgrund von Messungenauigkeit) und $\Delta \tau = \qty{1}{\percent}$ angenommen.}
    \label{tab:polarisation}
\end{table}
%
% ====
%
\begin{table}[ht]
    \centering
    \begin{tabular}{
        S[table-format=3.1(2)]          % tau
        S[table-format=4.0(2)]          % U
    }
    \toprule
    {$\tau$ / ms} &
    {$U$ / mV}\\
    \midrule
 2.00 \pm 0.02 & 943 \pm 50 \\
 4.00 \pm 0.04 & 831 \pm 50 \\
 6.00 \pm 0.06 & 743 \pm 50 \\
 8.00 \pm 0.08 & 668 \pm 50 \\
10.00 \pm 0.10 & 612 \pm 50 \\
12.00 \pm 0.12 & 531 \pm 50 \\
14.00 \pm 0.14 & 481 \pm 50 \\
16.00 \pm 0.16 & 437 \pm 50 \\
18.00 \pm 0.18 & 400 \pm 50 \\
20.00 \pm 0.20 & 356 \pm 50 \\
25.00 \pm 0.25 & 287 \pm 50 \\
30.00 \pm 0.30 & 225 \pm 50 \\
35.00 \pm 0.35 & 181 \pm 50 \\
40.00 \pm 0.40 & 150 \pm 50 \\
45.00 \pm 0.45 & 131 \pm 50 \\
50.00 \pm 0.50 & 116 \pm 50 \\
60.00 \pm 0.60 &  85 \pm 50 \\
    \bottomrule
    \end{tabular}
    \caption{Messwerte des Hahn-Echos. Für eine variable Wartezeit $\tau$ wird die Amplitude $U$ des Echo-Signals gemessen. Es wurden Fehler von $\Delta U = \qty{50}{\milli\volt}$ (aufgrund von Messungenauigkeit) und $\Delta \tau = \qty{1}{\percent}$ angenommen.}
    \label{tab:hahn}
\end{table}
%
% ====
%
\begin{table}[ht]
    \centering
    \begin{tabular}{
        S[table-format=3.1(2)]          % tau
        S[table-format=4.0(2)]          % U
    }
    \toprule
    {$\tau$ / ms} &
    {$U$ / mV}\\
    \midrule
  0.2 \pm 1.3 & 1117 \pm 56 \\
  6.8 \pm 1.3 & 836 \pm 42 \\
 13.6 \pm 1.3 & 686 \pm 34 \\
 20.2 \pm 1.3 & 580 \pm 29 \\
 27.2 \pm 1.3 & 486 \pm 24 \\
 34.0 \pm 1.3 & 430 \pm 21 \\
 40.8 \pm 1.3 & 380 \pm 19 \\
 47.6 \pm 1.3 & 330 \pm 16 \\
 54.4 \pm 1.3 & 280 \pm 14 \\
 61.2 \pm 1.3 & 255 \pm 13 \\
 68.0 \pm 1.3 & 230 \pm 11 \\
 74.8 \pm 1.3 & 205 \pm 10 \\
 81.8 \pm 1.3 & 186 \pm  9 \\
 88.2 \pm 1.3 & 167 \pm  8 \\
 95.0 \pm 1.3 & 136 \pm  7 \\
101.4 \pm 1.3 & 130 \pm  6 \\
108.4 \pm 1.3 & 130 \pm  6 \\
115.6 \pm 1.3 & 117 \pm  6 \\
122.6 \pm 1.3 & 99 \pm  5 \\
129.2 \pm 1.3 & 99 \pm  5 \\
    \bottomrule
    \end{tabular}
    \caption{Messwerte der Carr-Purcell-Sequenz. Für eine variable Wartezeit $\tau$ wird die Amplitude $U$ des Echo-Signals gemessen. Es wurden Fehler von $\Delta U = \qty{5}{\percent}$ (aufgrund der Ungenauigkeit des \enquote{peak-finder}-Programms) und $\Delta \tau = \qty{1}{\percent}$ angenommen.}
    \label{tab:carr}
\end{table}
%
% ====
%
\begin{table}[ht]
    \centering
    \begin{tabular}{
        S[table-format=3.1(2)]          % tau
        S[table-format=4.0(2)]          % U
    }
    \toprule
    {$\tau$ / ms} &
    {$U$ / mV}\\
    \midrule
  0.2 \pm 1.3 & 1112 \pm 56 \\
  6.6 \pm 1.3 &  850 \pm 43 \\
 13.6 \pm 1.3 &  675 \pm 34 \\
 20.4 \pm 1.3 &  550 \pm 28 \\
 27.2 \pm 1.3 &  500 \pm 25 \\
 34.2 \pm 1.3 &  431 \pm 22 \\
 40.8 \pm 1.3 &  393 \pm 20 \\
 47.6 \pm 1.3 &  356 \pm 18 \\
 54.4 \pm 1.3 &  300 \pm 15 \\
 61.6 \pm 1.3 &  256 \pm 13 \\
 68.0 \pm 1.3 &  250 \pm 13 \\
 75.2 \pm 1.3 &  218 \pm 11 \\
 81.4 \pm 1.3 &  212 \pm 11 \\
 88.4 \pm 1.3 &  200 \pm 10 \\
 94.8 \pm 1.3 &  175 \pm  9 \\
102.4 \pm 1.3 &  156 \pm  8 \\
108.6 \pm 1.3 &  156 \pm  8 \\
115.8 \pm 1.3 &  137 \pm  7 \\
129.6 \pm 1.3 &  125 \pm  6 \\
    \bottomrule
    \end{tabular}
    \caption{Messwerte der Meiboom-Gill-Sequenz. Für eine variable Wartezeit $\tau$ wird die Amplitude $U$ des Echo-Signals gemessen. Es wurden Fehler von $\Delta U = \qty{5}{\percent}$ (aufgrund der Ungenauigkeit des \enquote{peak-finder}-Programms) und $\Delta \tau = \qty{1}{\percent}$ angenommen.}
    \label{tab:meiboom}
\end{table}