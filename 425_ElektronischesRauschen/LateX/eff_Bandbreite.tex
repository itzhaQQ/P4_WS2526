\chapter{Effektive Bandbreite}

Im ersten Versuchsteil wird die effektive Bandbreite $\Delta f_{\text{eff}}$ des Bandpasses bestimmt, um Rauschprozesse
in realen Systemen zu analysieren. Ein Bandpass besteht aus einem Hochpass, der nur Frequenzen oberhalb einer bestimmten 
Grenzfrequenz durchlässt, und einem Tiefpass, der Frequenzen unterhalbeiner bestimmten Grenzfrequenz durchlässt. Zwischen 
diesen beiden Grenzfrequenzen liegt die sogenannte Durchlassbandbreite.
Um $\Delta f_{\text{eff}}$ zu bestimmen, wird die Verstärkung $G$ des Filters in Abhängigkeit von der Frequenz des 
Hochpasses $f_{\text{high}}$ und des Tiefpasses $f_{\text{low}}$ gemessen. Frequenzfilter können durch das sogenannte 
Übertragungsverhalten oder die Übertragungsfunktion charakterisiert werden. Diese ist definiert als das Verhältnis der 
Ausgangs- zur Eingangsgröße in Abhängigkeit von der Frequenz:

\begin{equation}
    G(f)= \frac{RMS_{out}}{RMS_{in}} = \frac{U_{out}}{U_{in}}.
    \label{eq:bandpass_verstaerkung}
\end{equation}
Dabei ist die RMS das quadratische Mittel der Spannung. 
Da es sich um eine Butherwoth-Filter 2.Ordnung handelt, sind die Übertragungsfunktion der 
Hoch- und Tiefpass folgendermaßen definiert: 
\begin{equation}
    G_{LP}(f) = \frac{1}{\sqrt{1+(f/f_{low})^4}}
\end{equation}
und 
\begin{equation}
    G_{HP}(f) = \frac{f}{f_{high}} \cdot \frac{1}{\sqrt{1+(f/f_{high})^4}}
\end{equation}
Die Gesamtverstärkung des Bandpasses ergibt sich aus dem Produkt der beiden Einzelverstärkungen: 
\begin{equation}
    G_{BP}(f) = G_{LP}(f) \cdot G_{HP}(f).
\end{equation}
Die effektive Bandbreite $\Delta f_{eff}$ des Bandpasses wird durch das Integral der Quadrat der
Übertragungsfunktion über alle Frequenzen definiert:
\begin{equation}
    \Delta f_{eff} = \int_0^{\infty} G_{BP}^2(f) df.
\end{equation}
\section{Durchführung}
Zur Analyse des Ansprechverhaltens der Frequenzfilter wurde das Verstärkungsprofil der \\ \enquote{\texttt{High-Level electronics}}-Box (HLE) gemessen. 
Die Aufbau ist in Abbildung \ref{fig:durchführung_bandpass} dargestellt.

\begin{figure}[h]
    \centering
    \includegraphics[width=0.7\textwidth]{figs/johnson_hle.png}
    \caption{Schaltplan zur Messung des Ansprechverhaltens der Frequenzfilter~\cite{praktikum4_atome}}\label{fig:eff_Bandbreite_hle}
    \label{fig:durchführung_bandpass}
\end{figure}

Um die Übertragungsfunktion zu bestimmen, wird ein Funktionsgenerator verwendet, der ein Sinussignal mit variierenden Amplituden erzeugt. Dieses Signal wird direkt auf Kanal 1 des Oszilloskops geleitet und parallel durch eine Kombination von Hoch- und Tiefpass gefiltert. Das gefilterte Signal wird auf Kanal 2 des Oszilloskops ausgegeben.
Anschließend werden verschiedene Frequenzen am Funktionsgenerator eingestellt und die Amplituden der Ein- und Ausgangssignale gemessen.
Nun Hoch- und tiepass sind fest auf $ \SI{100}{\kilo\hertz} $ und 
 $ \SI{10}{\kilo\hertz} $ dementsprechend eingestellt.
 Das Signal des Funktionsgenerators und das Ausgangssignal des Bandpasses werden auf dem Oszilloskops
 angezeigt. Die Amplituden der Signale werden für verschiedene Frequenzen des Funktionsgenerators notiert.
\section{Auswertung}
Der Beitrag der Übertragungsfunktion ist durch die Verstärkung $G$ gegeben und ist bei der vorliegende
Aufbau in Gleichung \ref{eq:bandpass_verstaerkung} definiert. Der Fehler der Verstärkung ergibt 
sich aus der Fehlerfortpflanzung der gemessenen Spannungen:
\begin{equation}
    \Delta G = \sqrt{\left(\frac{1}{U_{in}} \Delta U_{out}\right)^2 + \left(-\frac{U_{out}}{U_{in}^2} \Delta U_{in}\right)^2 }.
\end{equation}
Dabei ist eine Unsicherheit von $5\%$ der gemessenen Spannungen angenommen worden und die gemessenen Werten 
sind in \cref{tab:eff_Bandbreite}\footnote{Aufgrund der aufbauspezifischen Werte wurde in diesem Teil die gemessenen Werte zusammen mit der anderen Laborgruppe aufgezeichnet.} dargestellt. Diesen Daten werden in einem $G(f)$ gegen $log_{10}(f)$-Diagramm aufgetragen
und sind in Abbildung \ref{fig:eff_Bandbreite_plot} dargestellt.

\begin{figure}[h]
    \centering
    \includegraphics[width=0.7\textwidth]{plots/eff_Bandbreite.pdf}
    \caption{$G(f)$ als Funktion der Frequenz mit angepasste Funktion mit $\chi^2 = 122.75$}
    \label{fig:eff_Bandbreite_plot}
\end{figure}

 Bei der Anpassung der Messdaten an die Funktion $G(f)$ ergeben sich die Parameter mit folgenden Werten: 
\begin{align*}
    f_{low} &= (11.15 \pm 0.23) \si{\kilo\hertz} \\
    f_{high} &= ( 0.89 \pm 0.02) \si{\kilo\hertz}
\end{align*}
Mit diesen Werten lässt sich die effektive Bandbreite $\Delta f_{eff}$ des Bandpasses berechnen:
\begin{align}
    \Delta f_{eff} = \int_0^{\infty} G^2(f) df &= \int_0^{\infty} \left(\frac{f}{f_{high}} \cdot \frac{1}{\sqrt{1+(f/f_{high})^4}} \cdot \frac{1}{\sqrt{1+(f/f_{low})^4}}\right)^2 df\\
    \Delta f_{eff} &= \frac{\sqrt{2} \pi f_{low}^4}{4(f_{hoch}+f_{low})(f_{low}^2 + f_{high}^2)}\\
    \sigma_{\Delta f} &= \sqrt{
        \left(\frac{\partial \Delta f}{\partial f_l}\cdot \sigma_{f_l}\right)^2 +
        \left(\frac{\partial \Delta f}{\partial f_h}\sigma_{f_h}\right)^2
    }
\end{align}
Dieses Integral kann numerisch ausgewertet werden und ergibt:
\begin{equation*}
    \Delta f_{eff} \approx 11.39  \si{\kilo\hertz}.
\end{equation*}


Somit beträgt die effektive Bandbreite des Bandpasses:
\begin{equation*}
    \Delta f_{eff} = (11.39 \pm 0.46) \si{\kilo\hertz}.
\end{equation*}
Die Unsicherheit von $\Delta f_{eff}$ wird hauptsächlich durch die Unsicherheiten der Grenzfrequenzen
 $f_{low}$ und $f_{high}$ beeinflusst. Daraus ergibt sich eine Unsicherheit von etwa $0.099$.
Diese effektive Bandbreite ist entscheidend für die Analyse von Rauschprozessen in elektronischen Systemen, da sie die
Frequenzbereiche definiert, in denen das System Signale effektiv verarbeiten kann.

Aus den gemessenen Werten ergibt sich $\chi^2 = 122.75$, was zunächst auf eine nicht optimale
 Anpassung hindeutet. Dies liegt jedoch daran, dass die Messwerte bei kleinen und großen Frequenzen 
 stark von der idealen Filterkurve abweichen. Dies könnte daran liegen, dass die Filter bei diesen 
 Frequenzen nicht ideal arbeiten und zusätzliche Verluste oder Verzerrungen auftreten. Dies beeinflusst 
 jedoch nicht die Bestimmung der effektiven Bandbreite, da diese hauptsächlich durch die mittleren 
 Frequenzen bestimmt wird. Trotz des hohen $\chi^2$-Wertes liefert die Anpassung dennoch eine gute Näherung für die
Bestimmung der Grenzfrequenzen und somit der effektiven Bandbreite.
\begin{table}[h]
    \centering
    \begin{tabular}{
        S[table-format=8.0] 
        S[table-format=1.2(2)]
        S[table-format=1.2(2)]
        S[table-format=1.3(2)]}
    \toprule
    \multicolumn{1}{c}{$f$ / Hz}& 
    \multicolumn{1}{c}{$\mathrm{RMS}_{in}$ / V}& 
    \multicolumn{1}{c}{$\mathrm{RMS}_{out}$ / V}& 
    \multicolumn{1}{c}{$G(f)$}\\
    \midrule
2             & 7.51 \pm 0.38  & 0.21 \pm 0.01  & 0.028 \pm 0.002 \\
5             & 7.49 \pm 0.37  & 0.21 \pm 0.01  & 0.027 \pm 0.002 \\
8             & 7.50 \pm 0.38  & 0.21 \pm 0.01  & 0.028 \pm 0.002 \\
20            & 7.46 \pm 0.37  & 0.20 \pm 0.01  & 0.027 \pm 0.002 \\
50            & 7.44 \pm 0.37  & 0.21 \pm 0.01  & 0.028 \pm 0.002 \\
80            & 7.40 \pm 0.37  & 0.21 \pm 0.01  & 0.028 \pm 0.002 \\
200           & 7.12 \pm 0.36  & 0.35 \pm 0.02  & 0.049 \pm 0.003 \\
500           & 7.40 \pm 0.37  & 1.90 \pm 0.10  & 0.257 \pm 0.018 \\
800           & 7.48 \pm 0.37  & 4.00 \pm 0.20  & 0.535 \pm 0.038 \\
2000          & 7.11 \pm 0.36  & 7.22 \pm 0.36  & 1.015 \pm 0.072 \\
5000          & 7.47 \pm 0.37  & 7.27 \pm 0.36  & 0.973 \pm 0.069 \\
8000          & 7.46 \pm 0.37  & 6.30 \pm 0.32  & 0.845 \pm 0.060 \\
20000         & 7.43 \pm 0.37  & 1.79 \pm 0.09  & 0.241 \pm 0.017 \\
50000         & 7.42 \pm 0.37  & 0.34 \pm 0.02  & 0.046 \pm 0.003 \\
80000         & 7.41 \pm 0.37  & 0.23 \pm 0.01  & 0.031 \pm 0.002 \\
200000        & 7.33 \pm 0.37  & 0.20 \pm 0.01  & 0.028 \pm 0.002 \\
500000        & 7.27 \pm 0.36  & 0.27 \pm 0.01  & 0.038 \pm 0.003 \\
800000        & 7.13 \pm 0.36  & 0.91 \pm 0.05  & 0.128 \pm 0.009 \\
2000000       & 7.42 \pm 0.37  & 0.45 \pm 0.02  & 0.061 \pm 0.004 \\
5000000       & 7.75 \pm 0.39  & 0.21 \pm 0.01  & 0.027 \pm 0.002 \\
8000000       & 7.99 \pm 0.40  & 0.21 \pm 0.01  & 0.026 \pm 0.002 \\
10000000      & 6.70 \pm 0.34  & 0.21 \pm 0.01  & 0.031 \pm 0.002 \\
    \bottomrule
    \end{tabular}
    \caption{Experimentelle Verstärkung $G(f)$ in Abhängigkeit der Frequenz $f$.}
    \label{tab:eff_Bandbreite}
\end{table}