\chapter{Fazit}
Im Versuch diesem Versuch wurden zwei fundamentale Arten elektronischen Rauschens untersucht: das Johnson-Nyquist-Rauschen (thermisches Rauschen) und das Schrotrauschen. Beide Experimente dienten der experimentellen Bestimmung grundlegender Naturkonstanten:der Boltzmann-Konstanten $k_\mathrm{B}$ und der Elementarladung $e$.
% ----
\section{Effektive Bandbreite}
% ----
Aus dem Frequenzgang der HLE-Box wurde für den verwendeten Bandpass eine effektive Bandbreite von 
\begin{equation*}
    \Delta f_\text{eff} = \qty{11,39 \pm 0.46}{\kilo\hertz}
\end{equation*}
bestimmt. Trotz eines relativ hohen $\chi^2 = 122.75$ (bedingt durch Abweichungen bei extremen Frequenzen) beschreibt die Anpassung den relevanten Frequenzbereich hinreichend gut. Die ermittelte Bandbreite bildet eine zentrale Grundlage für die Auswertung der Rauschmessungen.

\section{Johnson-Rauschen}

Die Abhängigkeit der mittleren quadratischen Rauschspannung $\langle V_J^2 \rangle$ vom Widerstand bestätigte qualitativ die lineare Beziehung gemäß der Nyquist-Gleichung:
\[
\langle V^2 \rangle = 4k_\mathrm{B}TR\Delta f.
\]
Die experimentell bestimmten Werte der Boltzmann-Konstanten betrugen:
\[
\begin{aligned}
k_\mathrm{B}^{(\text{alle Daten})} &= 2.36(10) \times 10^{-23}\,\mathrm{J\,K^{-1}},\\
k_\mathrm{B}^{(-1\,\text{Punkt})} &= 1.14(5) \times 10^{-22}\,\mathrm{J\,K^{-1}},\\
k_\mathrm{B}^{(-2\,\text{Punkte})} &= 1.39(6) \times 10^{-22}\,\mathrm{J\,K^{-1}},
\end{aligned}
\]
wobei der Literaturwert 
\[
k_\mathrm{B}^{\text{lit}} = 1.380\,649 \times 10^{-23}\,\mathrm{J\,K^{-1}}
\]
ist~\cite{NIST}.\\
Die Abweichungen sind teils beträchtlich und können auf verschiedene Faktoren zurückgeführt werden:
\begin{itemize}
    \item nicht ideale Filtereigenschaften außerhalb des Durchlassbereichs,
    \item Temperaturunterschiede und Selbstaufheizung der Widerstände,
    \item externe Störquellen (z.\,B. akustische Kopplung der Verstärker),
    \item zu grobe oder ungleichmäßig verteilte Messpunkte bei der Bandbreitenvariation.
\end{itemize}

Insbesondere die kleineren Widerstände heizten sich stärker auf, wodurch der effektive Temperaturwert unterschätzt wurde und somit die berechneten $k_\mathrm{B}$-Werte zu groß ausfielen.

Die zweite Messreihe zur Bandbreitenabhängigkeit ergab:
\[
k_\mathrm{B} = 2.09(3) \times 10^{-24}\,\mathrm{J\,K^{-1}},
\]
was ebenfalls deutlich unter dem Literaturwert liegt. Hauptursache ist hier die geringe Anzahl unabhängiger Messpunkte und ein zu kleiner angenommener Fehler.

\section{Schrotrauschen}

Beim Schrotrauschen wurde der photoelektrisch erzeugte Strom einer Photodiode analysiert. 
Die lineare Abhängigkeit der Stromrauschleistung $\delta_i^2$ sowohl vom Gleichstrom $i_\mathrm{dc}$ als auch von der Bandbreite $\Delta f$ wurde bestätigt, entsprechend der Schottky-Beziehung:
\[
\delta_i^2 = 2e\, i_\mathrm{dc}\, \Delta f.
\]
Die experimentell bestimmten Werte für die Elementarladung lauten:
\[
\begin{aligned}
e_{\text{exp,}\,i_\mathrm{dc}} &= 1.89(40)\times 10^{-19}\,\mathrm{C},\\
e_{\text{exp,}\,\Delta f} &= 1.75(5)\times 10^{-19}\,\mathrm{C},
\end{aligned}
\]
im Vergleich zum Literaturwert
\[
e_\text{lit} = 1.602\,176\,634\times 10^{-19}\,\mathrm{C}.
\]
Damit ergeben sich relative Abweichungen von \SI{17.9}{\percent} bzw. \SI{7.9}{\percent}, also eine insgesamt sehr gute Übereinstimmung.

Die reduzierte Chi-Quadrat-Analyse ergab:
\begin{itemize}
    \item $\chi^2_\text{red} = 0.04$ für die Gleichstrommessung $\rightarrow$ überhöhte Fehlerabschätzung,
    \item $\chi^2_\text{red} = 0.8$ für die Bandbreitenmessung $\rightarrow$ gute Modellanpassung.
\end{itemize}

Somit konnte das Schrotrauschen erfolgreich nachgewiesen und die Elementarladung innerhalb weniger Prozent des Literaturwertes bestimmt werden.