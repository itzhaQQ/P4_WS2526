\chapter{Einleitung}
% ---
In diesem Versuch wird das elektronische Rauschen untersucht. Elektronisches Rauschen ist ein fundamentales
Phänomen in elektrischen Schaltungen und Systemen, das die Qualität und Präzision von Messungen beeinflussen 
kann. 
Das Rauschen entsteht letztlich durch die zufällige, thermische Bewegung der Elektronen in den Bauteilen. 
Diese natürliche \enquote{Unruhe} der Ladungsträger erzeugt unvermeidbare Schwankungen in Strom und Spannung.
Es gibt verschiedene Arten von Rauschen, darunter das thermische Rauschen, auch als Johnson-Nyquist-Rauschen 
bezeichnet, das durch die thermische Bewegung von Ladungsträgern in einem Widerstand verursacht wird.
Ein weiteres wichtiges Rauschphänomen ist das Schrotrauschen, auch als Schottky-Rauschen bekannt,
das in elektronischen Bauelementen wie Dioden und Transistoren auftritt.
Diese beiden Arten von Rauschen werden in diesem Versuch untersucht und analysiert, um zwei fundamentale 
Größen der Physik zu bestimmen: die Boltzmann-Konstante $k_B$ und die Elementarladung $e$~\cite{praktikum4_atome}\footnote{Sofern nicht anders angegeben werden Durchführung und Einstellungen aus dem \emph{Praktikum 4: Atome, Moleküle, kondensierte Materie}~\cite{praktikum4_atome} entnommen.}