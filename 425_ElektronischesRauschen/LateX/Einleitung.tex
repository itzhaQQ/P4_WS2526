\chapter{Einleitung}
% ---
In diesem Versuch werden zwei Arten von elektronischem Rauschen 
untersucht, ein fundamentales Phänomen in elektrischen Schaltungen und Systemen, das die Qualität und Präzision von Messungen beeinflussen kann.\par
Es gibt verschiedene Arten von Rauschen, darunter das 
thermische Rauschen, auch als Johnson-Nyquist-Rauschen 
bezeichnet, welches durch die thermische Bewegung von 
Ladungsträgern in einem Widerstand verursacht wird.\par
Ein weiteres Rauschphänomen ist das Schrotrauschen, auch als Schottky-Rauschen bekannt,
welches in elektronischen Bauelementen wie Dioden und 
Transistoren auftritt.\par
Diese beiden Arten von Rauschen werden in diesem Versuch untersucht und analysiert, um zwei fundamentale Größen der Physik zu bestimmen: die Boltzmann-Konstante $k_B$ und die Elementarladung $e$.\cite{praktikum4_atome}\footnote{Sofern nicht anders angegeben werden Durchführung und Einstellungen aus dem \emph{Praktikum 4: Atome, Moleküle, kondensierte Materie} \cite{praktikum4_atome} entnommen.}

