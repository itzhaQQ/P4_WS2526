\chapter{Schrotrauschen}
Schrotrauschen entsteht aufgrund der zufälligen statistischen Fluktuation diskreter elektrischer Ladungsträger. Er tritt auf, sobald unkorrelierte Elektronen eine Potentialbarriere (z.B. Photodiode) überwinden. Dieser Prozess ist statistischer Natur und folgt der Poisson-Verteilung.\\
In diesem Versuchsteli wird ein solcher Elektronenstrom durch den Photoeffekt erzeugt. Eine Glühlampe emittiert diskrete Photonen, welche auf eine Photodiode treffen und dort Elektronen rauslösen, welche wiederum Stromfluktuationen erzeugen.\\
Diese lassen sich über die Schottky-Formel quantifizieren:
\begin{equation}
    \overline{\delta_i^2} = 2ei_{\mathrm{dc}}\Delta f
\end{equation}
wobei $\overline{\delta_i^2}$ die mittlere quadratische Stromschwankungen, $e$ die Elementarladung, $i_{dc}$ den Mittelwert des Gleichstroms (Photostrom) und $\Delta f$ die effektive Bandbreite des Messsystems sind.\\
Die effektive Bandbreite erhält man aus dem Frequenzgang der verwendeten Filter:
\begin{equation}
    \Delta f = \int_{0}^{\infty} G(f)^2\,\mathrm{d}f
\end{equation}
mit der Verstärkungsfunktion des verwendeten Tiefpassfilters:
\begin{equation}
    G(f) = \left( 1 + \left( \frac{f}{f_l} \right)^4 \right)
\end{equation}
Der Versuchsaufbau zur Erzeugung des Photostroms ist in \cref{fig:schrot_schema} dargestellt. Die bipolare Spannungsquelle versorgt die Glühlampe, welche Photonen emittiert und die Photodiode bestrahlt die wiederum einen Gleichstrom $i_{\mathrm{dc}}$ erzeugt.
\begin{figure}[h]
    \centering
    \includegraphics[width=0.7\linewidth]{figs/schrot_schaltung.png}
    \caption{Schaltschema für die LLE-Box zur erzeugung des Photostroms mittels Glühlampe und Photodiode~\cite{praktikum4_atome}}\label{fig:schrot_schema}
\end{figure}
Die Photodiode ist über einen invertierenden Verstärker mit Rückkopplungswiderstand $R_f = \qty{10}{\kilo\ohm}$ in der LLE-Box verschaltet. Der erzeugt Photostrom erzeugt eine proportionale Spannung:
\begin{equation}
    V_{\mathrm{Monitor}} = -R_f\cdot i_{\mathrm{dc}}, \quad \Delta V_{\mathrm{Monitor}} = R_f\cdot \Delta i_{\mathrm{dc}}
\end{equation}
Das vollständige Schaltschema ist in \cref{fig:schrot_aufbau} zu sehen.
\begin{figure}[h]
    \centering
    \subcaptionbox{LLE-Box: Schaltung zur Verstärkung des Photostroms}{\includegraphics[keepaspectratio, width=0.49\linewidth]{figs/schrot_lle.png}}
    \subcaptionbox{HLE-Box: Schaltung zur Messung des Schrotrauschens}{\includegraphics[keepaspectratio, width=0.49\linewidth]{figs/schrot_hle.png}}
    \caption{Schaltung zur Messung des Schrotrauschens}\label{fig:schrot_aufbau}
\end{figure}
Das Signal wird nach der LLE-Box weitergeleitet zur HLE-Box (\cref{fig:schrot_aufbau}b). Der verbaute Multiplier wird im $AxA$-Modus betrieben und berechnet den quadratischen Mittelwert über die eingestellte Zeitkonstante von $\qty{1}{\second}$ gemäß \cref{equ:multiplier}.\\
Das Ausgangssignal der HLE-Box $V_{\mathrm{Meter}}$ ist proportional zur mittleren Stromschwankung:
\begin{equation}
    \overline{\delta_i^2} = \frac{\overline{V_{\mathrm{Meter}}(t)}\cdot \qty{10}{\volt}}{(100\cdot G_2\cdot R_f)^2},\quad \Delta\overline{\delta_i^2} = \frac{\Delta\overline{V_{\mathrm{Meter}}(t)}\cdot \qty{10}{\volt}}{(100\cdot G_2\cdot R_f)^2}
\end{equation}
Zur Messung des Schrotrauschens wird der Tiefpass der HLE-Box auf eine Grenzfreuqenz $f_{\mathrm{gr}} = \qty{100}{\kilo\hertz}$ eingestellt und alle Schalter auf AC eingestellt. Die Verstärkung $G_2$ wird so gewählt, dass das Signal $\overline{V_{\mathrm{Meter}}}$ Werte im Bereich von \SIrange{0.6}{1.2}{\volt} anzeigt, um im linearen Bereich zu bleiben.\\
Das Rauschen wurde nun auf einem Oszilloskop sichtbar gemacht.
\begin{figure}[h]
    \centering
    \includegraphics[keepaspectratio, width=0.6\linewidth]{images/IMG_0956.png}
    \caption{Darstellung des Schrotrauschens auf dem Oszilloskop $V$/div $=\qty{5}{\volt}$; $T$/div $=\qty{4}{\micro\second}$}
\end{figure}
% ----
\section{Untergrundbeiträge}
% ----
Um sicherzustellen, dass die Messungen nicht von anderen Rauschquellen abstammen (Johnson-Rauschen am Widerstand, Offset-Spannung am Verstärker, Feedback dem DMM am Monitor-Ausang des Vorverstärkers) muss eine Untergrundmessung durchgeführt werden.\\
Die Lampenspannung wird hierfür auf Null gesetzt und der Regler $G_2$ auf $1000$ eingestellt.\\
Gemessen wurde eine Augangsspannung:
\begin{equation*}
    V_{\mathrm{Meter}, 0} = \qty{31.7\pm 0.1}{\milli\volt}
\end{equation*} 
woraus sich eine mittlere Rauschleistung von 
\begin{equation*}
    \overline{\delta_{i,\mathrm{offset}}^2} = \qty{3.17(1)e-19}{\volt}
\end{equation*}
Der gemessene Fehler hängt jedoch nicht nur von der HLE-Box, sondern auch vom Vorverstärker ab. Deswegen wird eine weitere Spannungsmessung bei abgeschalteter Glühlampe am Monitor Ausgang der LLE-Box durchgeführt.\\
Hier beträgt $V_{\mathrm{Monitor}, 0}$ jedoch Null und hat somit keinen Beitrag zu Gleichstrom $i_\mathrm{dc}$. Auch wurde der Einfluss des DMM am Monitor-Ausgang des Vorverstärker untersucht, in dem er mehrmals aus- und eingesteckt wurde, jedoch konnte auch hier kein Beitrag notiert und somit der Einfluss vernachlässigt werden.
% ----
\section{Abhängigkeit des Schrotrauschens vom Gleichstrom}
% ----
Um die Abhängigkeit des Schrotrauschens vom Gleichstrom $i_{\mathrm{dc}}$ zu Untersuchen werden die Ausgangsspannungen $V_{\mathrm{Monitor}}$ und $V_\mathrm{Meter}$ für verschiedene Glühbirnenspannungen vermessen (\cref{tab:schrot_strom}).
\begin{table}[h!]
    \centering
    \begin{tabular}{
        S[table-format=4.0]
        S[table-format=2.2(1)]
        S[table-format=1.2(1)]
        S[table-format=1.2(1)]
        S[table-format=1.2(2)]
    }
    \toprule
    \multicolumn{1}{c}{$G$} &
    \multicolumn{1}{c}{$V_\mathrm{Monitor}$ / V} &
    \multicolumn{1}{c}{$V_\mathrm{Meter}$ / V} &
    \multicolumn{1}{c}{$i_\mathrm{dc}$ / A} &
    \multicolumn{1}{c}{$\overline{\delta_i^2}$ / A$^2$} \\
    \midrule
    1500 & 1,01  \pm 0,05  & 0,92 \pm 0,05 & 0,10 \pm 0,01 & 0,38 \pm 0,02 \\
    1000 & 2,05  \pm 0,05  & 0,84 \pm 0,05 & 0,20 \pm 0,01 & 0,81 \pm 0,05 \\
     800 & 3,04  \pm 0,05  & 0,79 \pm 0,05 & 0,30 \pm 0,01 & 1,21 \pm 0,08 \\
     800 & 4,07  \pm 0,05  & 1,11 \pm 0,05 & 0,41 \pm 0,01 & 1,70 \pm 0,08 \\
     600 & 5,02  \pm 0,05  & 0,74 \pm 0,05 & 0,50 \pm 0,01 & 2,03 \pm 0,14 \\
     600 & 6,11  \pm 0,05  & 0,92 \pm 0,05 & 0,61 \pm 0,01 & 2,51 \pm 0,14 \\
     600 & 7,02  \pm 0,05  & 1,06 \pm 0,05 & 0,70 \pm 0,01 & 2,91 \pm 0,14 \\
     500 & 8,12  \pm 0,05  & 0,85 \pm 0,05 & 0,81 \pm 0,01 & 3,36 \pm 0,20 \\
     500 & 9,02  \pm 0,05  & 0,94 \pm 0,05 & 0,90 \pm 0,01 & 3,72 \pm 0,20 \\
     500 & 10,12 \pm 0,05  & 1,06 \pm 0,05 & 1,01 \pm 0,01 & 4,20 \pm 0,20 \\
     400 & 11,21 \pm 0,05  & 0,75 \pm 0,05 & 1,12 \pm 0,01 & 4,68 \pm 0,31 \\
     400 & 12,11 \pm 0,05  & 0,81 \pm 0,05 & 1,21 \pm 0,01 & 5,03 \pm 0,31 \\
     400 & 13,11 \pm 0,05  & 0,88 \pm 0,05 & 1,31 \pm 0,01 & 5,47 \pm 0,31 \\
    \bottomrule
    \end{tabular}
    \caption{Messwerte zur Abhängigkeit des Schrotrauschens von dem Gleichstrom: Verstärkung $G$, Monitorspannung $V_\mathrm{Monitor}$, gemessene Spannung $V_\mathrm{Meter}$, Gleichstrom $i_\mathrm{dc}$ und normierte Stromrauschleistung $\delta^2$.}\label{tab:schrot_strom}
\end{table}
Zur Berücksichtigung des Untergrundrauschens werden die Rauschschwankungen um den Offset korrigiert:
\begin{equation}
    \overline{\delta_i^2} = \frac{\overline{V_{\mathrm{Meter}}(t)}\cdot \qty{10}{\volt}}{(100\cdot G_2 R_f)^2} - \overline{\delta_{i, \mathrm{offset}}^2},\quad\Delta\overline{\delta_i^2} = \sqrt{\left(\frac{\overline{\Delta V_{\mathrm{Meter}}(t)}\cdot \qty{10}{\volt}}{(100\cdot G_2 R_f)^2}\right)^2 - \left(\Delta\overline{\delta_{i, \mathrm{offset}}^2}\right)^2} 
\end{equation}
und in \cref{fig:schrot_strom} aufgetragen.
\begin{figure}
    \centering
    \includegraphics[width=0.8\linewidth]{plots/schrot.png}
    \caption{Darstellung der mittleren quadratischen Stromschwankung in Abhängigkeit vom Gleichstrom}\label{fig:schrot_strom}
\end{figure}
Der Zusammenhang lässt sich dirch lieare Regression der Form $y(x) = mx+b$ beschreiben und liefert:
\begin{itemize}
    \item $m = \qty{4.20(9)e-14}{\ampere}$
    \item $b = \qty{-4.61(254)e-19}{\ampere\squared}$
    \item $\chi_{\mathrm{red}}^2 = 0.04$
\end{itemize}
was für ein sehr gutes Anpassungsmodell spricht. Dies lässt sich durch Betrachtung der (normierten) Residuen bestätigen, da diese nah um Null herum verteilt sind.\\
% ----
\section{Abhängigkeit des Schrotrauschens von der Bandbreite}
% ---
Um die Abhängigkeit des Schrotrauschens von der Bandbreite und somit seine \enquote{Weißheit}(also konstante Rauschleistung über die Bandbreite) zu untersuchen wird nun der Gleichstrom $i_\mathrm{dc}$ konstant gehalten, in dem $V_{\mathrm{Monitor}}=\qty{10.02\pm 0.05}{\volt}$ eingestellt wurde. Die Grenzfrequenz des Tiefpasses $f_l$ wird nun variiert, wobei geachtet wurde, dass die Verstärkung $G_2$ eine Ausgangsspannung der HLE-Box zwischen \SIrange{0.6}{1.2}{\volt} ermöglicht. Der Widerstand der LLE-Box wurde auf $R_f = \qty{10}{\kilo\ohm}$ eingestellt.
\begin{table}[h!]
    \centering
    \begin{tabular}{
        S[table-format=6.0]
        S[table-format=4.0]
        S[table-format=1.2]
        S[table-format=6.2]
        S[table-format=+1.3(3)]
    }
    \toprule
    \multicolumn{1}{c}{$f_l$ / Hz} &
    \multicolumn{1}{c}{$G_2$} &
    \multicolumn{1}{c}{$V_\mathrm{Meter}$ / V} &
    \multicolumn{1}{c}{$\Delta f$ / Hz} &
    \multicolumn{1}{c}{$\overline{\delta_i^2}$ / A$^2\cdot 10^{-17}$} \\
    \midrule
   100000 & 4000 & 0,70 \pm 0,05 & 111072,07 & 4,325 \pm 0,313 \\
    33000 &  800 & 0,85 \pm 0,05 &  36653,78 & 1,293 \pm 0,078 \\
    10000 & 1500 & 0,86 \pm 0,05 &  11107,21 & 0,352 \pm 0,022 \\
     3300 & 3000 & 1,10 \pm 0,05 &   3665,38 & 0,0903 \pm 0,0056 \\
     1000 & 4000 & 0,60 \pm 0,05 &   1110,72 & 0,0059 \pm 0,0031 \\
      330 & 8000 & 0,70 \pm 0,05 &    366,54 & -0,0207 \pm 0,0008 \\
    \bottomrule
    \end{tabular}
    \caption{Messwerte zur Abhängigkeit des Schrotrauschens von der Bandbreite: untere Grenzfrequenz $f_l$, Verstärkung $G_2$, gemessene Spannung $V_\mathrm{Meter}$, Bandbreite $\Delta f$ und mittlere Stromrauschleistung $\overline{\delta_i^2}$.}\label{tab:schrot_bandbreite}
\end{table}
